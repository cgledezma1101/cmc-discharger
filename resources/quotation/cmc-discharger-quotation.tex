\documentclass[9pt, letterpaper, oneside]{report}

\usepackage[margin=2cm]{geometry}
\usepackage[spanish]{babel}
\usepackage[utf8]{inputenc}

\begin{document}
  \begin{center}
    \underline{Cotización de sistema de altas para el Centro Médico de Caracas}
  \end{center}

  \underline{Exposición de motivos}

  A continuación se presenta la cotización para un sistema con los
  requerimientos solicitados. El mismo responderá a la necesidad de automatizar
  el proceso de dar de alta a un paciente, una vez que éste haya cumplido
  con el tiempo previsto en la clínica. Además el sistema recogerá estadísticas
  respecto al tiempo que pasó el paciente en cada una de las etapas del alta, lo
  que permitirá tomar las medidas necesarias para agilizar cada proceso.

  El costo en honorarios profesionales para la realización de este sistema será
  de 100 BsF por hora invertida. Luego, la corrección de cualquier error que los
  usuarios detecten en la aplicación se hará de manera gratuita, como garantía
  por la labor realizada. Se entenderá por error cualquier funcionalidad que
  presente el sistema que no sea acorde a lo que se estableció en las reuniones
  que se realizaron antes de la instalación. Cualquier modificación o extensión
  de alguna funcionalidad que se solicite después de la
  instalación del sistema, o que no se encuentre contemplada en este
  presupuesto, deberá ser analizada y cotizada en el momento en el que se
  solicite.

  Se empezará a realizar el trabajo cuando se reciba un adelanto del 50\% del
  costo del mismo. El resto deberá ser cancelado una vez que se finalice la
  instalación de la aplicación.

  \vspace{3mm}

  \underline{Presupuesto}

  \vspace{3mm}

  \begin{tabular}{p{0.5\textwidth} | c | c}
    Actividad & Horas de dedicación & Costo (BsF) \\
    \hline
      Documentación e investigación acerca de las herramientas a utilizar &
      3 &
      300 \\
    \hline
      Realización del diseño conceptual de la aplicación &
      2 &
      200 \\
    \hline
      Implementación de una base de datos de acuerdo al diseño realizado &
      2 &
      200 \\
    \hline
      Creación de un conjunto de direcciones que permitan la interacción con el
        sistema &
      1 &
      100 \\
    \hline
      Implementación del sistema de autenticación &
      2 &
      200 \\
    \hline
      Implementación de una pantalla que contenga una de lista, donde se puedan
        visualizar los pacientes que están en el proceso de alta &
      5 &
      500 \\
    \hline
      Implementación de las llamadas a Javascript necesarias para interactuar de
        forma asíncrona con las listas &
      5 &
      500 \\
    \hline
      Implementación de una pantalla donde se puedan visualizar las
        estadísticas &
      4 &
      400 \\
    \hline
      Implementación de los controladores que introducirán la lógica a las
        pantallas desarrolladas &
      8 &
      800 \\
    \hline
      Depuración y pruebas &
      5 &
      500 \\
    \hline
      Instalación de la aplicación en las premisas &
      6 &
      600 \\
    \hline
      Total &
      43 &
      4300 \\
  \end{tabular}

  \vspace{10mm}

  Atentamente,

  \vspace{20mm}

  Carlos Ledezma

\end{document}
